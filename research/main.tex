\documentclass[twocolumn]{article}
\usepackage[utf8]{inputenc}
\usepackage{graphicx}

\title{American Sign Language\\
	(ASL)\\
	Fingerspelling Detection}
\author{Shoaib Mohammed}
\date{01 June 2021}


\begin{document}

\begin{titlepage}
\maketitle
\thispagestyle{empty}
\end{titlepage}

\begin{abstract}
	Approximately 70 million people around the world are deaf-mute. While 
	translation services have become easily accessible for about 100 
	languages, sign language is still an area that has not been explored. 
	Our goal is to detect \& translate the letters of American Sign Language 
	(ASL) in real-time.
\end{abstract}

\section{This is First Level Heading}
This is the first level heading.

\subsection{This is Second Level Heading}
This is the second level heading.

\subsubsection{This is Third Level Heading}
This is the third level heading.

\section{Introduction}

\subsection{Background}
According to the Communication Service for the Deaf (CSD) [1], there are 360 
million deaf people worldwide. Another report by the World Health 
Organization (WHO) [2] bumps up the number to 466 million people suffering 
from disabling hearing loss. Future projections estimate 630 million people by 
2030 and over 900 million people by 2050. But even in this age of technology 
and communication, we are yet to see a universal translation system that helps 
bridge the gap between people that can and cannot speak. The goal of this 
project is to detect and accurately translate the letters in American Sign 
Language (ASL).

\subsection{Scope \& Limitations}
The bottom line is that there is no universal sign language. For instance, the 
British Sign Language (BSL) differs by a great margin from the American Sign 
Language (ASL). Generally speaking, a person in the US can understand spoken 
English in the UK but this is not the case with sign language.

% (TODO: add a reference to the statistic for more than 200 sign languages)
It is interesting to note that there are more than 200 sign languages that are 
used across the world. However, if an accurate model was developed to 
recognize a sign language, in our case, ASL, the same methodology could be 
applied to recognize other sign languages. Perhaps the major limitation is 
classifying each sign from the sheer corpus of signs in ASL (TODO: add a 
reference showing why it’s a sheer corpus). However, since our focus is 
classifying the alphabets alone, our range is limited to 24 alphabets since we 
exclude the alphabets J \& Z since these require motion.

\section{Literature Survey}

\subsection{Population Statistics}
A combined study in [3] estimates there were more than 250,000 deaf people and 
as many as 500,000 people who used ASL in 1972. Over the years, this number 
has been on the rise.

The Survey of Income and Program Participation (SIPP) collects data for the US 
population. From the research of Ross E. Mitchell [3] in the year 2006, fewer 
than 1 in 20 Americans or 10,000,000 people suffered from hard of hearing and 
close to 1,000,000 were classified as functionally deaf.

\end{document}
